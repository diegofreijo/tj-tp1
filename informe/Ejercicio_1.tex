% Ejercicio 1
\subsection{Clasificaci'on del juego}
\italica{Demonios} recibe las siguientes calificaciones:
\begin{itemize}
\item Este juego es sin azar, dado que las jugadas de un jugador son el resultado de solamente sus desiciones.
\item Bipersonal, ya que intervienen dos jugadores (cada uno controla un demonio).
\item Es finito debido que el tablero contiene finita cantidad de casilleros y en cada jugada la cantidad de casillas disponibles para saltar disminuye estrictamente.
\item De informaci'on perfecta, ya que en todo momento ambos jugadores conocen las posibles jugadas de ambos jugadores, y pueden ver que jugada realiza el oponente.
\item Es de informaci'on completa, ya que tanto las estrategias como los pagos son conocidos por ambos jugadores al iniciar el juego.
\item No es un juego cooperativo, como as'i tampoco de movida simult'nea, sino secuencial.
\item Es de suma cero, porque los pagos asociados son:
	\begin{itemize}
	\item 1, si el jugador gana.
	\item -1, si el jugador pierde.
	\item 0, si se llegara a empatar.
	\end{itemize}
\end{itemize}


% Ejercicio 2
\subsection{Estrategia}
Presentamos e implementamos una estrategia que decidimos bautizar Demonio \italica{Pacifico}. Principalmente, trata de sobrevivir lo mas posible, sin preocuparse en molestar al oponente. 

El algoritmo para decidir a que casillero saltar es el siguiente:





El c'odigo fuente de la estrategia y 'arbitro se pueden encontrar en la carpeta \italica{src}, junto a la implementaci'on de un torneo (una sucesi'on de jugadas variando el tama'no del tablero). Los binarios se encuentran en el directorio \italica{bin}.


%\end{enumerate}
