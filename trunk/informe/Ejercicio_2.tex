\begin{enumerate}
\item Clasificaci'on del juego:

\begin{itemize}
\item Este juego es sin azar.

\item Bipersonal, ya que intervienen dos jugadores, uno juega con las fichas blancas, el otro con las negras.

\item Finitud

\item Informaci'on perfecta/imperfecta  (CLASE 3)

\item Es de Informaci'on Completa ya que tanto las estrategias como los pagos son conocidos por ambos jugadores al iniciar el juego.

\item No es un juego Cooperativo, como as'i tampoco de Movida Simult'nea, sino Secuencial.

\item Suma Cero. Ya que los pagos asociados son:
\begin{itemize}
\item 1, si el jugador gana.
\item -1, si el jugador pierde.
\item 0, si se llegara a empatar.
\end{itemize}
En un principio no hay empate pero, como es (POTENCIALMENTE) infinito, podr'ia forzarse con una regla del tipo \it{cuando se repita una cierta cantidad de veces una jugada se puede pedir el empate}.
\end{itemize}

\item Grafo con todos los estados del juego:

\item Valor del juego. Mejores estrategias.
En un principio parece dificil determinarse a ciencia cierta si el juego tiene valor y m'as a'un, cual es dicho valor.

Analisando el juego vimos que el \it{Pong hau k'\ i} es:

\begin{itemize}
\item de dos jugadores
\item de informaci'on perfecta
\end{itemize}

Luego, aplicando el teorema de Kuhn, podemos asegurar que el juego \negrita{tiene valor}.



%Luego, si aplicamos la regla:
%$$cuando\ se\ repita\ una\ cierta\ cantidad\ de\ veces\ una\ jugada\ se\ puede\ pedir\ el\ empate$$
%Estar'iamos forzando a que sea finito, ya que no se podr'ian repetir las jugadas por siempre. Ahora tenemos un nuevo juego, llamemoslo \it{Pong hau k'\ i finito}

%Notar que el juego \it{Pong hau k'\ i} se reduce en \it{Pong hau k'\ i finito}, por lo tanto sus valores son iguales.


\end{enumerate}
